\documentclass[]{iac}
\usepackage[strings]{underscore}
\usepackage{bm} % used for bold math symbols
\usepackage{enumitem}
\usepackage{glossaries}

\DeclareMathOperator{\E}{E}
\DeclareMathOperator{\prob}{p}
\DeclareMathOperator{\tr}{tr}

\newcommand{\etalia}{\textit{et al.}}
\newcommand*{\vectornorm}[1]{\left\|#1\right\|}
\newcommand*\rfrac[2]{{{}^{#1}\!/_{#2}}} % running fraction with slash - requires math mode.
\newcommand*\T{\mathsf{T}}

\makeglossaries

\newglossaryentry{COTS}{%
name={COTS},%
description={Commercial Off The Shelf}%
}
\newglossaryentry{IREC}{%
name={IREC},%
description={Intercollegiate Rocket Competition}%
}
\newglossaryentry{UWB}{%
name={UWB},%
description={Ultrawide Band}%
}

\newglossaryentry{FEM}{%
name={FEM},%
description={Finite Element Method}%
}


\newglossaryentry{CNC}{%
name={CNC},%
description={Computer Numerical Control}%
}


\newglossaryentry{AGL}{%
name={AGL},%
description={Above Ground Level}%
}


\newglossaryentry{IB}{%
name={IB},%
description={Interface Board }%
}

\newglossaryentry{PB}{%
name={PB},%
description={Power Board}%
}
\newglossaryentry{SAW}{%
name={SAW},%
description={Surface Acoustic Waves }%
}
\newglossaryentry{AD}{%
name={AD},%
description={Avionic Drawer }%
}

\newglossaryentry{AR}{%
name={AR},%
description={Avionic Rack }%
}

\newglossaryentry{GNSS}{%
name={GNSS},%
description={Global Navigation Satellite System}%
}
\newglossaryentry{RTK}{%
name={RTK},%
description={Real Time Kinematic }%
}
\newglossaryentry{IMU}{%
name={IMU},%
description={Inertial Measurement Unit }%
}







\glsaddall



\begin{document}

\IACpaperyear{17}
\IACpapernumber{D9.2.8}
\IACconference{65}
\IAClocation{Toronto, Canada TODO UPDATE}
\IACcopyrightA{2017}{the authors}

\title{Technology demonstrator of a rocket carrying a deployable fleet of autonomous gliders}

%\author{Main~Author\\Affiliation, Country, email address\and
%		Co-Author\\Affiliation, Country, email address}
\IACauthor{Patrick Spieler}{Swiss Federal Institute of Technology, Lausanne, Switzerland, patrick.spieler@alumni.epfl.ch}


\IACauthor{Elena Sorina Lupu, Dalmir Hasic, Michael Spieler, Michael Pellet, Hassan Arif, Cyril Baumann, Emilio Lozano, Quentin Talon}{Swiss Federal Institute of Technology, Lausanne, Switzerland}

\IACauthor{Oliver Kirchhoff, Laurent Jung}{Swiss Federal Institute of Technology, Zurich, Switzerland}

\IACauthor{Stephane Teste}{John Hopkins University, United States of America}

\IACauthor{Christian Cardinaux}{Western Switzerland University of Applied Sciences - HEIG-VD, Switzerland}

\IACauthor{Dr. Anton Ivanov, Prof. Dario Floreano, Dr. Stefano Mintchev}{Swiss Federal Institute of Technology, Lausanne, Switzerland}





\abstract{%Rewrite: Dalmir, Sorina
%NOTE: Many of the structure of paper goes to the end of introduction
%The Intercollegiate Rocket Competition (IREC) aims to gather students from across the world to design and build a rocket that reaches an altitude of 3km or 10km, carrying a 4kg of payload. As part of this competition, the Team Duster, formed by students from Swiss universities developed a rocket with a payload targeting an apogee of 3km. \\ 
%Firstly, the rocket flying to 3km will be presented, along with its design and manufacturing processes. The rocket then follows a dual-event recovery process. Firstly, the drogue (small) parachute is deployed, reducing the speed of the rocket to about 30 m/s. At the same time the separation of the nosecone occurs thereby releasing the payload. After the rocket descends to an altitude of 457 meters, the second (main) parachute is deployed, reducing the speed to 6m/s. Throughout the launch, separation, and landing phases, we constantly received telemetry (information) provided by the custom-designed avionic components placed in the nosecone of the rocket. To ensure that the deployment occurs at predefined altitudes, the decision was made to use 2 redundant systems for altitude measurement -independent of the avionics placed in the recovery bay. The trajectory of the rocket was simulated in 3 varying environments, including our own developed simulator. The rocket made its first flight towards the end of March 2017 in Switzerland where its flight data was compared with the simulation data. Based on this data, the necessary trajectory corrections were performed in order to improve the competition Flight - which occured in mid-June. \\
%Secondly, an innovative payload that flew in the rocket will also be presented. A fleet of 3 gliders are deployed from the rocket at apogee using a release based on a spring mechanism. Equipped with an autopilot, differential GPS, and control surfaces using servomotors, the gliders autonomously fly in formation. Eventually, they land at a fixed point on the ground. The gliders have. During the flight, the gliders transmit information back to the groundstation and are tracked using custom-made ground station.  Potential video-recording of the flight will be investigated in the future.
%The results of the intermediary flight (end-of-March) as well as the competition flight - both in terms of rocket trajectory and flight of the gliders - is included in the paper, along with further recommendations for a more advanced technology demonstration in the future.

The Intercollegiate Rocket Competition (IREC) aims to gather students across the world to design and build a rocket that reaches an altitude of 10,000 ft (3 km) or 30,000 ft (9 km), carrying 8.8 lb (4kg) of payload. As a part of this competition, team DUSTER, formed by students from 3 Swiss universities designed and built a rocket with a payload targeting an apogee of 3km. The rocket is launched by using COTS solid motor. It follows a dual-event recovery process with first parachute deployment at apogee and second (main) parachute deployment at 457 meters.
Inside the rocket a payload of 3,9 kg is placed and that payload consists of a custom built autonomous glider ejected at apogee, which descends to a fixed point on the ground independent of the rocket.   Custom avionics inside the nosecone log and transmit information back to the ground, control the ejection of the nosecone and the payload deployment.
To estimate the apogee under variety of conditions and motor configurations 3 separate simulators are used with one being developed by the team as a part of this project. One of the major concerns during the development is the stability - both static and dynamic, which influences the design and has large impact on the performance of the rocket.
The technology behind both the rocket and the payload together with concepts for autonomous fleet of gliders is presented and elaborated on in this paper. Design process and manufacturing, along with tests done and issues faced over the course of the project are also presented along with the solutions and recommendations for the future.
}

% Dalmir, can you add the abreviations? 
\maketitle
%Check for number of pages allowed, if any
\section{Introduction}

%Spaceport America Cup (formerly entitled Intercollegiate Rocket Competition (IREC)) gatheres students all over the world to participate at the biggest rocket challenge. Students are launching solid, liquid, and hybrid rockets to target altitudes of 10,000 feet (3048 meters) and 30,000 feet (9144 meters), carrying 8,8lb (3.9 kg) of payload.

%Team DUSTER, representing several Swiss universities, entered this contest to build a solid motor rocket flying to an apogee of 10,000 ft (3048 meters). The project received an honourable place 8$^th$, out of 83 teams.

%This paper contains the work developed by the team of students between November 2016 and June 2017. The first part of the article discusses the rocket design, manufacturing and flight data. The second part of the article reviews the payload placed inside the rocket. For this year, the team built a glider that was deployed at 10,000 ft and flew back to the ground. For the future, the aim is to develop a fleet of gliders that is deployed from the rocket at apogee and flies in formation back to the ground. As part of this article, the fleet of gliders will be only conceptually discussed.

Spaceport America Cup (formerly entitled Intercollegiate Rocket Competition (IREC)) gathers students from all over the world to participate at one of the biggest rocket competitions in the world. Students are launching solid, liquid, and hybrid rockets to target altitudes of 10,000 feet (3048 meters) and 30,000 feet (9144 meters), carrying 8,8lb (3.9 kg) of payload. The teams are scored according to the flight performance on the competition day, technical implementation, the quality of the report and research done as a part of the project.
Team DUSTER, representing several Swiss universities, entered this contest to build a solid motor rocket flying to an apogee of 10,000 ft (3048 meters).
The main objective of the rocket is to achieve an apogee of 10,000 ft exactly. Deviations from this apogee result in point loss in the flight performance category.  The rocket is designed to follow a dual event recovery with 2 separation points - one at the apogee and the other at 457 meters. At the apogee 3 events occur: separation alongside the body of the rocket where the parachutes are ejected, ejection of the nosecone from the rocket and payload separation.

Recovery subsystem is responsible for obtaining 
a precise apogee and ensuring the deployment of the parachutes. This system consists of 2 parachutes and 2 redundant recovery avionics instruments which are used to estimate the apogee and trigger the separation of the rocket. After the separation occurs at the apogee, both parachutes are ejected from the body of the rocket and one is opened , the so-called drogue, reducing the speed of the rocket to about 30 m/s. The main parachute remains in the bag for the first part of the descend. At the same time the nosecone is also separated from the rocket and the glider is ejected. From this point, the rocket consisting of 3 parts (2 body parts and the nosecone) and the glider descend independently of each other. At the altitude of 457 meters, the second (main) parachute opens, reducing the speed to 6m/s. Throughout the launch, separation and landing phases telemetry information, provided by the custom-designed avionic components placed in the nosecone of the rocket, are being constantly logged and part of them also sent to the ground station.

In the model flown at this year's competition no control mechanisms were in place and also no air brakes were installed meaning once the rocket is launched the trajectory cannot be influenced. This increased the importance of simulation, as the only way to determine the altitude the rocket can obtain with a chosen motor and assuming certain weather conditions.  The trajectory of the rocket was simulated in 3 varying environments, including our own developed simulator  \cite{sim_hassan}.

% Dalmir, we should cite the simulator of Hassan, here it is the link: https://github.com/irec-duster/RORO/blob/master/simulation/simulator%20docs/IdentificationandControlofaHighPowerRocket-HA.pdf


The second main component of the project is the payload. The payload is a glider equipped with an autopilot, differential GPS, and control surfaces using servomotors. Its objective is to land at a fixed point on the ground and transmit data from the sensors to the ground station. In this year's model only one glider had been constructed with a wing span of approximately 70 cm. The concept of multiple gliders flying in a formation is presented as well with a possibility of being implemented in the following years.
The results of the intermediary flight (end-of-March) as well as the competition flight - both in terms of rocket trajectory and flight of the gliders - is included in the paper, along with further recommendations for a more advanced technology demonstration in the future.

The structure of this paper is as follows: in Section 2 the rocket is presented together with 4 main subsystems: structure, avionics, propulsion and recovery. Flight tests together with stability considerations are outlined in Section \ref{subsection:flightTests}.  Section \ref{section:gliders}  covers the design and technology used in a glider and presents a concept of a fleet of gliders. Navigation and control, one of the main challenges in glider design, are elaborated in detail in Section \ref{subsection:navcontrol}. The last part contains the conclusion and the outlook.


\section{Rocket}


The rocket, entitled RORO I, is a 8 feet (2.45 meters) rocket propelled by a M-class solid motor. The main requirements of the rocket are presented in Table \ref{table:se_topLevelR}.

\begin{table}[h!]
\centering
\begin{tabular}{|p{0.9\columnwidth}|}
\hline
    The rocket shall achieve an apogee of 10,000 ft (3048 meters).  \\ \hline
    The rocket shall have a static margin between 1 and 2 body-calibers. \\ \hline
    The rocket shall carry a COTS barometric pressure altimeter with on-board storage as primary data source for altitude reporting.  \\ \hline
    The launch vehicle shall follow a "dual-event" recovery. \\ \hline
    The rocket shall carry a minimum mass of 8.8 lb (4 kg) of payload. \\ \hline
    The rocket shall eject its nosecone at apogee. \\ \hline
    The rocket shall release a glider from the payload section at 10 seconds after apogee. \\ \hline

\end{tabular}
\caption{Top Level Requirements for the rocket}
\label{table:se_topLevelR}
\end{table}


\subsection{Design and Manufacturing}


The rocket is divided into 3 main sub-assemblies (Figure \ref{f:rocket_adnoted}):

\begin{enumerate}[noitemsep]
    \item The nosecone - carrying avionics, the locking system for the noseocone and its ejection system
    \item Upper Body - carrying the payload, the parachutes and the recovery electronics
    \item The Lower Body - containing logging avionics and the motor
\end{enumerate}
The concept of the rocket was well-thought to be easy to integrate and robust, considering the time constrains. The rocket separats in the middle, between the upper and lower body. This approach is very common in High Power Rocketry (HPR) and is considered a less risky approach.

\begin{figure}[h!]
\centering
\includegraphics[width=0.5\textwidth]{img/rocket_sw_annotated.jpg}
\caption{The rocket with its main components}
\label{f:rocket_adnoted}
\end{figure}


The length as well as the diameter of the rocket were chosen in such a way to accommodate the payload which had a dimension constraint imposed by the competition (needed to be Cubesat standard) and the other subsystems.

%Next, the manufacturing of each subsystem of the rocket is discussed.

Before considering the stability in the following sections, we discuss manufacturing process of each part of the rocket.



\paragraph{Structure}
\hfill \break
The main body of the rocket is a COTS phenolic tube, reinforced with two layers of 245 g/m$^2$ will weaved carbon fiber. The reinforcement rational is determined using FEA (Finite Element Analysis), and tested during two rocket flights.
The structure contains also a coupler tube, that is made out of phenolic tube reinforced inside again with two 245 g/m$^2$ layers carbon fiber.
The rocket has two structural bulkheads, one in the lower body and one in the upper body to which the parachute cords are attached. The bulkheads are made out of two 15mm plywood plates reinforced with 4 layers of 245 g/m$^2$ carbon fibre on each side.  The bulkheads are the structurally critical elements as they transfer loads from the rocket body to the subsystems. They need to withstand both the accelerations from the launch and the parachute opening shock. 
The two bulkheads are subject to a peak load of 10kN from the parachute opening shock. An FEM analysis taking into account the  ECSS-E-HB-32-21A \cite{ecss} revealed that an additional reinforcement of the bulkheads with carbon fibre is required to sustain the loads. The analysis of the bonding to the rocket body reveals sufficient strength to sustain the opening shock, findings also validated in practice.

% Dalmir, can you city the ECSS? http://www.ecss.nl/wp-content/uploads/handbooks/ecss-e-hb/ECSS-E-HB-32-21A20March2011.pdf

% Dalmir, can you city the ECSS? http://www.ecss.nl/wp-content/uploads/handbooks/ecss-e-hb/ECSS-E-HB-32-21A20March2011.pdf

% todo put the static caliber
%\paragraph{Fins}
%\hfill \break
%    The fins are sized for 1.1 calibers static stability %Reference to static/dynamic stability
 %   and were manufactured out of wood and carbon fiber, as it can be seen in Figure \ref{f:fins}. Firstly, the wood was cut at CNC (Figure \ref{f:fins} a). The inside of the fins was made out of balsa wood (Figure \ref{f:fins} b), in order to decrease the weight for higher resonance frequencies. Afterwards, on each side of each fin, 3 layers [span, chord, span] of 140 g/m$^2$ unidirectional carbon fiber were applied, to increase stiffness (Figure \ref{f:fins} c). The fins were attached to the motor tube using high temperature epoxy, reinforced with carbon fiber.

%    \begin{figure}[h!]
  %      \centering
     %   \includegraphics[width=0.5\textwidth]{img/fins.jpg}
        %\caption{a) Wood at CNC b) Final wood-made fins c) Carbon fiber manufacturing d) Final version of the fins.}
        %\label{f:fins}
    %\end{figure}
\paragraph{Fins}
\hfill \break
    The fins are sized for 1.1 calibers static stability %Reference to static/dynamic stability
    and are manufactured out of wood and carbon fiber. The process is shown in Figure \ref{f:fins}. Firstly, the wood is cut at CNC (Figure \ref{f:fins} a). The inside of the fins is made out of balsa wood (Figure \ref{f:fins} b), in order to decrease the weight for higher resonance frequencies. Afterwards, on each side of each fin, 3 layers [span, chord, span] of 140 g/m$^2$ unidirectional carbon fiber are applied, to increase stiffness (Figure \ref{f:fins} c). The fins are attached to the motor tube using high temperature epoxy, reinforced with carbon fiber.
    \begin{figure}[h!]
        \centering
        \includegraphics[width=0.5\textwidth]{img/fins.jpg}
        \caption{a) Wood at CNC b) Final wood-made fins c) Carbon fiber manufacturing d) Final version of the fins.}
        \label{f:fins}
    \end{figure}

\paragraph{Motor tube}
\hfill \break
The motor tube consists of a Commercial-Off-The-Shelf phenolic tube. The fins are glued with 3M DP760 high-temperature epoxy to the motor tube. The tube is centered to the outer rocket body tube by six 4mm plywood centering rings distributed in equally along the motor tube.
In front of the fins, there is a 12mm CNC-cut plywood ring from which 12 M3 threaded rods connect to the thrust plate. These help holding the motor inside the rocket body during parachute opening shock.
The entire assembly can be seen in Figure \ref{f:reinforcement} c. The fins are fixed to the outer structure using ribbons of carbon fiber both on the inside and the outside of the tube, as it can be seen in Figure \ref{f:reinforcement} a,b.

  \begin{figure}[h!]
\centering
\includegraphics[width=0.5\textwidth]{img/fins_glue.jpg}
\caption{a, b) Reinforcement of the fins c)Motor tube assembly, with fins and centering rings}
\label{f:reinforcement}
\end{figure}


\paragraph{Motor case}
\hfill \break
The motor case, which is a RMS-98/7680 from Aerotech is held by a 98mm retainer from Aeropack on a custom made laser-cut aluminum thrust plate. The thrust-plate pushes directly on the fins which go through the body tube to the motor tube. The thrust-plate is used to attach the motor and to distribute the loads when parachute opens.
 The retainer \& thrust-plate assembly can be seen in Figure \ref{f:motor_retainer_2}.
\begin{figure}[h!]
        \centering
        \includegraphics[width=0.3\textwidth]{img/motor_retainer.jpg}
        \caption{Motor Retainer and the Aluminum plate}
        \label{f:motor_retainer_2}
    \end{figure}


\paragraph{Payload}
\hfill \break
In the lower body of the rocket, there is 1U of payload 
consisting of 4kg of tungsten and a board with sensors used to track the lower body motion and shocks, referred to as the Logging Electronics. A Gopro and a PCB cameras are also placed to film the parachute deployment and the outside.
The Active Payload bay, placed in the Upper Body, consists of a 4U plywood box reinforced with glass fibre to withstand the loads. Inside the 4U wood box, a glider is mounted on a rail. Shortly after apogee, the glider will be ejected from the rocket using a spring. A more detailed description of the glider will be detailed in Section \ref{section:gliders}




\subsection{Recovery}
This subsystem implements a dual-event recovery concept of operations (CONOPS) with an initial deployment event at apogee and a main deployment event at 457m (1500ft) AGL. Figure \ref{f:recovery_conops} illustrates the recovery CONOPS.
\begin{figure}[h!]
 	\centering
        \includegraphics[width=0.45\textwidth]{img/recovery_conops_schema.png}
        \caption{Recovery CONOPS}
        \label{f:recovery_conops}
 \end{figure}

During the initial deployment, the rocket is separated and a drogue parachute is released to stabilize the attitude and reduce descent rate to 23-46 m/s (75-150ft/s).At 457m above ground level (AGL) the main deployment takes place where the main parachute is released from the parachute bag to reduce the rocket's descent rate to less than 9m/s (30ft/s) to prevent excessive damage upon impact. Recovery Avionics system triggers the deployment events when the preprogrammed deployment conditions are met by firing a pyrotechnical charge.

\paragraph{Initial Deployment System}
\hfill \break
The initial deployment is achieved by creating a pressure in the parachute bay and thus forcing the two rocket bodies to separate. The pressure is generated by puncturing a 23ml CO2 cartridge by a 0.2 ml black powder charge. The charge forces a puncture piston inside the cartridge seal to release the gas. This system is a COTS solution from Tinder Rocketry Recovery Solutions.
There are two redundant CO2 deployment systems each connected to two igniters. 2 redundant recovery electronic components are in charge of triggering the system and each of the two can trigger both systems. Recovery avionics is outlined later in this section.

\paragraph{Main Deployment System}
\hfill \break
The main parachute is stowed in the bag below the drogue parachute. The main parachute is held together by a wire which is cut at the programmed altitude. Then the load of the rocket pulls the main parachute out of the parachute bag. The setup is illustrated in Figure \ref{f:recovery_main_deployment}. The wire is cut by a shearing piston which is forced through the wire by a 0.1ml black powder charge.Two wire cutters are installed for redundancy, one per recovery electronics.

\begin{figure}[h!]
 	\centering
        \includegraphics[width=0.45\textwidth]{img/recovery_main_deployment.jpg}
        \caption{Main Parachute Deployment}
        \label{f:recovery_main_deployment}
 \end{figure}

 \paragraph{Recovery Avionics}
 \hfill \break
The Recovery Avionics is designed for maximal reliability and features several levels of redundancy which is shown in Figure \ref{f:recovery_avionics_schema}. As required by the the competition rules, two redundant electronic systems are used, both of which are different COTS solutions.
The primary electronics is the AltimaxG3 from Rocketronics which features barometer and accelerometer sensors to estimate the altitude of the rocket. It uses a Kalman filter to estimate acceleration, speed and altitude of the rocket. This is a more robust solution especially when pressure fluctuations can be expected at a high velocity.
The backup electronics is the Raven3 from Featherweight Altimeters.
The primary electronics is programmed to detect the apogee and fire the two CO2 charges one after the other with a delay of 0.5 seconds. Then during descent it monitors air pressure until the main parachute deployment altitude is reached and initiates the deployment by using the wire cutter.
 The backup electronics is programmed as a timer to trigger the initial deployment after the predicted time to apogee from simulations. During descent it also detects the target altitude using a pressure sensor to trigger the main deployment

  \begin{figure}[h!]
 	\centering
        \includegraphics[width=0.5\textwidth]{img/recovery_avionics_schema.png}
        \caption{Recovery Avionics Schema}
        \label{f:recovery_avionics_schema}
 \end{figure}
  

\paragraph{Recovery Bay} 
 The recovery structure is made out of laser cut and milled plywood glued together with epoxy. It is screwed onto a circular bulkhead glued into the upper rocket tube. The connection to the parachute bay is made as airtight as possible by sealing holes with glue and using rubber washers.
Figure \ref{f:recovery_bay} shows the assembled recovery bay.
 \begin{figure}[h!]
 	\centering
        \includegraphics[width=0.45\textwidth]{img/recovery_bay.jpg}
        \caption{Recovery Bay}
        \label{f:recovery_bay}
 \end{figure}
 
 
\subsection{Avionics}
The avionics consist of 2 PCBs stacked together. The first one, referred to as the \textbf{Interface Board (IB)} shown in Figure \ref{f:avionics_ib}, is equipped with:
\begin{itemize}[noitemsep]
    \item an absolute pressure sensor for altitude measurement
    \item 2 differential pressure sensors for the Pitot tube
    \item an IMU
    \item a microcontroller (STM32F4 of STMicroelectronics)
    \item a GNSS receiver
    \item Xbee for telemetry downlink to the ground station
    \item 64 MB of flash memory for data logging
    \item interfaces such as 2 RS232 ports and USB port
\end{itemize}

The second PCB, the so called \textbf{Power Board (PB)}  shown in Figure \ref{f:avionics_pb}, is equipped with a voltage regulator that power all the nosecone avionics and up to 6 servo outputs.

All electronic components used for the avionics are automotive graded or better to ensure a proper working of the system under vibration and high temperature. Moreover, all the components were selected to be easily checked under microscope after soldering (e.g. no BGA chip are used as they usually require X-ray inspection).
The decision to split the avionics into 2 PCBs is made to overcome the limited space available and to avoid issues due to EMC. The buck converter on the PB can potentially generate a lot of EM perturbations, thus the sensitive parts such as the GNSS are placed away from it, namely on the IB.
As the rocket body is made of carbon fibre, the only RF transparent part of the rocket is the nosecone (made of fibre glass). Therefore, all the electronics components were integrated and placed there.

 \begin{figure}[h!]
 	\centering
        \includegraphics[width=0.45\textwidth]{img/AV_FIG_IB_top.jpg}
          \includegraphics[width=0.45\textwidth]{img/AV_FIG_IB_bottom.jpg}
        \caption{Interface Board (IB) Top and Bottom)}
        \label{f:avionics_ib}
 \end{figure}
 
  \begin{figure}[h!]
 	\centering
        \includegraphics[width=0.45\textwidth]{img/AV_FIG_PB_top.jpg}
          \includegraphics[width=0.45\textwidth]{img/AV_FIG_PB_bottom.jpg}
        \caption{Power Board (PB) Top and Bottom}
        \label{f:avionics_pb}
 \end{figure}

\paragraph{Sensors}
 \hfill \break
The Pitot tube uses 2 differential pressure sensors of Honeywell in parallel with different pressure range (0 to 6890Pa and 0 to 103350 Pa). This allows reduction of uncertainties at low speed (uncertainty of 6 m/s at 10 m/s) and ensures good performances at higher speed (uncertainty of 2m/s at 280m/s).
The IMU (3 Space High-G of Yost Labs) which is directly soldered on the IB can measure acceleration up to 24g and rate of turn up to 2000/s.
By using its integrated data possessing unit, it can directly output orientation (Euler angles, quaternions or rotation matrix) as well as acceleration, rate of turn, magnetic field and velocity increments with a data rate of up to 250 Hz.

The GNSS chip is a NEO-M8T of uBlox equipped with a SAW (surface acoustic waves) filter. As the telemetry antenna (XBee) is close to the 1st GNSS antenna, 2nd SAW filter (SF1186G of muRata) is placed just before the RF input in order to ensure an optimum operation of the GNSS (e.g. avoid jamming due to the XBee frequencies).There are 2 GNSS active antennas. One is placed at the back of the nose cone and is oriented toward the back of the rocket and the other one is placed at the front of the nose cone and is oriented toward the front of the rocket. The 2 GNSS active antennas are controlled via an RF switch on the IB. The first GNSS antenna is used during ascend of the rocket while the second one is used during the descent, after the deployment of the drogue parachute and the nosecone ejection.

\paragraph{Telemetry Downlink}
 \hfill \break
For the telemetry downlink, a 900MHz XBee (XB9X-DMUS-001 of Digi International) is placed on the IB and connected to a 2.1dBi omni antenna. Second XBee is directly connected to the ground station which uses a high gain Yagi antenna (23dB) that is manually oriented towards the rocked during flight. This configuration ensures a good power margin (45dBm) above the sensitivity level of the receiver when at maximum theoretical distance (5 km). The XBee on the IB is fixed with 2 screws and is easily accessible and replaceable in order to replace it according to the country of operation (868 MHz for CH and 900 MHz for the US). Using XBee at 2.4GHz (which is legal for both CH and US) was not an option as they don't have a sufficient communication range

\paragraph{Actuators}
 \hfill \break
The PB is equipped with 6 servo outputs, each allowing to control a high power servo. These outputs could be used in the future to implement an active control system of the rocket. In RORO I, 2 of these outputs are used. One is used to control the mechanism which ejects the nosecone while the other is used to control the release of the payload (the glider). 

\paragraph{Power Supply}
 \hfill \break
A high power buck converter from Texas Instrument is implemented on the PB. This device allows to power the avionics with 5 VDC with up to 15 A with 2S to 4S (6V to 17V) LiPo batteries. The PB is also equipped with 2 ideal diodes IC that allow to use 2 batteries in parallel for redundancy. In the case of RORO I, 2 3S, 1800 mAh LiPo are used. 

\paragraph{Software}
 \hfill \break
The software implemented in the avionics for RORO I mainly integrates the following functionalities: Data logging in the flash memory of the measurements performed by the sensors, detection of the launch of the rocket (with a threshold on the acceleration data from the IMU), sending the data of the GNSS (position) to the ground station via the XBee module, detection of the apogee of the rocket in order to open the nosecone 5 seconds after this event using a servo and switching to the 2nd GNSS antenna after the deployment of the nosecone. Five seconds after the deployment of the nosecone, second servo is trigged which allows the deployment of the payload.

\subsubsection{Mechanical Design and Manufacturing}
\paragraph{Mechanical Integration of the Avionics in the Nosecone}
 \hfill \break
The mechanical structure is designed with the idea of having the avionics easily accessible on the field. The 2 avionics PCBs are staked between 2 CNC machined plates of 3 mm tick plywood and fixed together with PCB spacers. On each plywood plate, gliding structures made out on plywood are glued. This assembly, called the avionic drawer (AD), can be easily slided into the avionic rack (AR) which is a structure made of several parts of CNC machined 6 mm tick plywood fixed to the control panel. To secure the AD, the aluminum GNSS ground plane of the 1st GNSS antenna is fixed on the top of it with nuts to two M5 threaded rods. These 2 threaded rods pass trough the avionic rack and are fixed to the control panel. On the control panel side, a hook is fixed at the end of each threated rod. These hooks are used to fix the rope that link the nosecone to the upper body bulkhead. The structure made of the control panel on which the AR together with the AD is fixed can be slided into the nosecone and be fixed with 6 M3 screws to a plywood crown glued with epoxy to the nosecone. The Avionics Bay inside the nosecone is depicted in Figure \ref{f:avionics_bay}.

  \begin{figure}[h!]
 	\centering
        \includegraphics[width=0.45\textwidth]{img/AV_FIG_CAD_nosecone.jpg}
        \caption{Avionics Bay inside the Nosecone}
        \label{f:avionics_bay}
 \end{figure}

\paragraph{Nosecone Deployment System}
 \hfill \break
\textit{Note: in the following paragraph, everything noted in parenthesis refers to the Figure \ref{f:av_deployment_sys}}

In order to get the satellite fix for the GNSS after apogee, the nosecone needs to be deployed. Moreover, this makes an opening in the upper body that allows to deploy the payload. So, to ensure the deployment of the nosecone, an ejection system was designed and manufactured that take into account the fact that the
main part of the volume below the nosecone is occupied by the payload.

The designed and manufactured system consist of 2 aluminum U-profiles (6.) fixed to the bulkhead (2.) of the upper body (1.). At the upper end of the u-profiles, a hook is fixed (7.) that is used by the looking system to secure the nosecone (5.). A gliding structure (3.) made out of 2 bigger aluminum U-profiles joined together with 4 arcs cut out of a phenolic tube can slide into the upper body.
In the smaller U-profiles (6.), a traction spring (4.) is fixed just bellow the hook (7.), the other end of the spring is fixed on the gliding structure (3.). So, when the nosecone is put in place, the upper part of the gliding structure comes in contact with the lower end of the nosecone and, thus, the gliding structure is pushed in the upper body which loads the 2 springs. When fully loaded, the 2 spring produce a total force of c.a. 100N. When the nosecone is released (thanks to the locking system) the gliding structure pushes it away from the upper body.

  \begin{figure}[h!]
 	\centering
        \includegraphics[width=0.45\textwidth]{img/AV_FIG_CAD_depl_sys_1.jpg}
        \caption{Deployment System of the Nosecone. LEFT: In place on the upper body. RIGHT: after deployment. (for clarity, payload is not shown)}
        \label{f:av_deployment_sys}
 \end{figure}

\paragraph{Nosecone Locking System}
 \hfill \break
\textit{Note: in the following paragraph, everything noted in parenthesis refers to the Figure \ref{f:av_locking_sys}}.

To secure the nosecone in place during the ascend of the rocket, a locking system was designed and manufactured. This system consist out of 2 steel pins of 6 mm diameter (2. and 3.) loaded with a compressive spring (4.) that can slide in a structure (1.) fixed to the control panel of the nosecone. An ?inverted? came wheel (5.) controlled with a servo that is fixed to the AR is used to move the 2 pins forth and back. When locked, the end of the 2 pins goes in a hook (7.) fixed to each outer rail (6.) of the
deployment system. When unlocked, the steel rods are retracted and comes out of the hooks. 
This configuration, with an inverted came wheel, allows to unlock the nosecone from the outside even if there is a failure of the servo or the avionics. Indeed, the axis of the 2 locking steel rods is aligned with the 2 holes (8.) made in the nosecone for Pitot tube?s static pressure measurement, thus, the locking system can be unlocked using 2 screwdriver to push the pins inside the nosecone

  \begin{figure}[h!]
 	\centering
        \includegraphics[width=0.45\textwidth]{img/AV_FIG_CAD_lockingsystem_22.jpg}
        \caption{Locking System of the Nosecone. Top: locked, Bottom: unlocked}
        \label{f:av_locking_sys}
 \end{figure}




\section{Gliders}
\label{section:gliders}

\subsection{General Overview}

Each team can choose an arbitrary payload to design and build. The team decided to develop long-term a set of gliders that are, once deployed from the rocket, flying back to the ground in formation. 


For this year, the team prepared a technology demonstrator consisting of only one glider deployed at apogee. This can be extrapolated for a fleet of 7 gliders. The challenge of building more gliders lies in fitting them into a constrained space in the rocket. 

\paragraph{Motivation for choosing the payload}
\hfill \break
Rovers such as Curiosity or Spirit have studied Martian terrain and sent back scientific data which may answer questions regarding the origin of life. However, in-situ atmospheric measurements of Mars on longer distances are missing. In order to solve this need, gliders, balloons, or powered planes should overfly Mars and land in areas where rovers cannot. Several projects are proposed by NASA in these directions such as the Preliminary Research Aerodynamic Design to Land on Mars (Prandtl-m) Airplane which aims to be released from a 3U CubeSat and do a 1-hour descent onto the surface of Mars.

Inspired by the Prandtl-m project, our team aimed to design and build an autonomous glider for the Spaceport America Cup competition, and learn more about the flight dynamics and control of such a complex system.

% ref
% @misc{mars,
%   title = {Could This Become the First Mars Airplane?},
%   howpublished = {\url{https://www.nasa.gov/centers/armstrong/features/mars_airplane.html}},
%   note = {Accessed: 2016-07-06}
% }

\begin{table}[h!]
\centering
\begin{tabular}{|p{0.9\columnwidth}|}
\hline
    The payload shall weight 8.8 lb (3.9 kg). \\ \hline
    The payload shall consist of a glider and ballast.  \\ \hline
    The glider shall be deployed at apogee. \\ \hline
    The glider shall deploy its wings passively once ejected from the rocket. \\ \hline
    The glider shall be equipped with an RTK (Real Time Kinematic) GPS used for navigation. \\ \hline
    The glider shall be equipped with a Commercial-Off-The-Shelve autopilot and a pitot tube \\ \hline
    The glider's battery life shall be 1.5 hours. \\ \hline
\end{tabular}
\caption{Top Level Requirements for the payload}
\label{table:se_topLevelR}
\end{table}


\subsection{Design and Manufacturing of ONE glider-SORINA}
%Pictures from report here

Due to space constraints, the glider was chosen to be a flying wing with the wings folded in front. XFLR simulations experiments with different airfoils, wing span, wing sweep were performed until the acceptable flight parameters were obtained. The final airfoil is a MH45, with improved (3\%) reflex.

The plane parameters are presented in Table X.


The optimal flight data is presented in Table Y.


\subsection{Design and Manufacturing of a fleet of gliders-SORINA}
%Concept TBD in solidworks 

\subsection{Navigation and Control of the fleet of gliders}
\label{subsection:navcontrol}
%Ultrawide beacons 
%Formation
%Check for alternative options, instead of GPS




\section{Results}

The rocket RORO I was flown twice. The first flight was a test flight in Switzerland and the second flight was the competition flight at Spaceport America Cup.

On board of the rocket there were an Inertial Measurement Unit (IMU) and a barometer that logged data to an SD card.
This data has been analyzed for both of the flights.

\subsection{Rocket Test Flight}

The test flight was performed in Kaltbrunn, Switzerland to a target apogee of 565m using an Aerotech K1275 motor and a lift-off mass of 18.3 kg. The apogee was limited to 600m due to proximity to buildings.

The rocket flight was successful, to an apogee of 502m, with two anomalies:
\begin{itemize}
    \item The main parachute opened at apogee instead of at 200m above ground
    \item The rocket oscillated at a high angle of attack
\end{itemize}

The main chute opening cause has been identified from th high quality video of the optical tracking done by ETHZ.
In conclusion, the parachute opening at apogee has been caused from the shock of the separated rocket stretching out the shock cord to the drogue parachute, where also the main chute bag was attached.
This shock pulled the main chute cords out of the bag, which then pulled out the entire main chute.

This design problem has been fixed by securing the main chute cords in the bag until the chute is released.

The second anomaly, the oscillation and high angle of attack, required an analysis of the IMU data as the cause of the high angle of attack could not be identified from video footage.

\subsubsection{IMU Data Analysis}

In the following, the z-axis is the roll axis and points down along the rocket and the x and y-axes are pitch and yaw respectively.

The IMU on board the rocket was am MPU-6000 MEMS 3-axis gyroscope and 3-axis accelerometer from InvenSense.

The gyroscope measures angular rate which has to be integrated to obtain the attitude of the rocket.
To reduce attitude drift due to high zero-rate bias of these MEMS gyroscopes, the bias was estimated by taking the average output while the rocket was on the launch rail from 30 seconds before liftoff to 5 seconds before liftoff.
This should give a reasonably precise attitude for the approximately 10 seconds to apogee.

The accelerometer bias could not be calibrated on the launch pad, as the precise inclination of the launch-rail was not known.
Therefore the accelerometer was calibrated after the flight for scale and bias by sequentially placing the IMU on each of the six faces up.
The bias for an axis is obtained by taking the mean between the averaged positive one $g$ acceleration and averaged negative one $g$ acceleration.
The scale for the axis is obtained by taking the half of the difference between the positive and negative one $g$ acceleration.
\begin{align}
    bias_i &= \frac{a^{+g}_i + a^{-g}_i}{2} \\
    scale_i &= \frac{a^{+g}_i - a^{-g}_i}{2g}
\end{align}


After sensor calibration, the first step was to integrate the angular rate to obtain the attitude of the rocket.
The attitude throughout the flight was visualized in 3D.
In the animation the tilting and subsequent oscillation of the rocket was well visible, but the source of the disturbance was not evident.


The next step was to correct the calibrated IMU accelerations for centripetal and tangential accelerations due to angular rate and angular acceleration, as the IMU is not placed at the center of mass. This gives us the acceleration at the center of mass:
\begin{equation}
    \bm{a}_{CM} = \bm{a}_{IMU} - \bm{a}_{tangential} - \bm{a}_{centripedal}
\end{equation}
\begin{equation}
\begin{split}
    \bm{a}_{tangential} &= \bm{\dot\omega} \times \bm{r}_{IMU} \\
    \bm{a}_{centripedal} &= - |\bm{\omega}_{tangential}|^2  \bm{r}_{IMU} \\
    \bm{\omega}_{tangential} &= \bm{\omega} - \frac{\bm{\omega} \cdot \bm{r}_{IMU}}{\bm{r}_{IMU}^2}\bm{r}_{IMU}
\end{split}
\end{equation}
where $\bm{r_{IMU}}$ is the position of the IMU with respect to the center of mass

The center of mass location is obtained from the Solidworks CAD (which was experimentally verified for the lift-off configuration) and is linearly interpolated from lift-off configuration to burn-out configuration for the time of the motor burn. This is good enough, as the motor burn curve is quite constant.

This acceleration at the center of mass was then used to compute the side-forces acting on the rocket using the known mass. Again the mass is interpolated during the burn as for the center of mass.

With the assumption that the side forces are aerodynamic, they should act at the center of pressure, which can be analytically determined.
This results in a moment which was calculated using \ref{eq:m_aero}.
\begin{equation}
\label{eq:m_aero}
\begin{split}
    M^{aero}_x &= (z_{CP} - z_{CM}) (-F_y) \\
    M^{aero}_y &= (z_{CP} - z_{CM}) F_x \\
\end{split}
\end{equation}

This aerodynamic moment was then compared with the actual moment that was acting on the rocket.
The actual moment was obtained from the angular rates and the inertia by use of the Euler equations for rigid body dynamics. For an equal pitch and yaw inertia these equations are:
\begin{equation}
\begin{split}
    M_x &= I_{xy} \dot\omega_x + (I_z-I_{xy}) \omega_y \omega_z \\
    M_y &= I_{xy} \dot\omega_y + (I_{xy}-I_z) \omega_z \omega_x \\
    M_z &= I_{z} \dot\omega_z
\end{split}
\end{equation}


The comparison between aerodynamic model based moments and actual moments is show in figure \ref{fig:test_flight_graph}.
\begin{figure}[h!]
    \centering
        \includegraphics[width=0.5\textwidth]{img/test_flight_graph.png}
        \caption{Comparison of predicted aerodynamic moments (M_aero_xy) and actual moments (M_xyz). The duration of the motor burn is clearly visible in the z acceleration.}
        \label{fig:test_flight_graph}
 \end{figure}

The following observations are made:
\begin{itemize}
\item The moment after burnout and reduction of oscillations to reasonable angles of attack matches well with aerodynamic prediction. This validates our analytic center of pressure determination.
\item The moment after burnout but at high angles of attack is slightly lower than predicted by the aerodynamic model. This is to be expected, as the model does not include the cylindrical body lift which increases with angle of attack, moving the center of pressure forward. A more forward center of pressure would result in the smaller moments that were seen at high angle of attack.
\item During the burn there is a permanent, about 20Nm mismatch between the aerodynamic prediction and the actual moments
\end{itemize}

The first two points validated the aerodynamic model.
They also showed that there was no torque acting on the rocket other the the expected aerodynamic restoring torques at non-zero angle of attack.

The third point gives an important clue to the origin of the disturbance. For the duration of the motor burn, there was a moment that is not of aerodynamic origin.
The best explanation is that this moment was caused by a misalignment of the motor thrust with respect to the center of gravity.

A side accelerometer sensitivity to forward acceleration can be ruled out, as the the side acceleration is around zero on the while accelerating on the launch rail.
The same applies for the gyroscope that is close to zero until leaving the launch rail.

There are several possible causes for the thrust misalignment.
\begin{itemize}
    \item The center of gravity is not centered in the rocket
    \item The rocket motor is not centered in the rocket
    \item The rocket motor was not aligned with the rocket body
    \item The rocket body bent under load which place the CG out of the line of thrust
    \item The rocket motor was aligned, but produced a side-thrust.
\end{itemize}

Careful inspection of the rocket 



\subsection{Rocket Competition Flight}



\subsection{Glider Test Flights}

TODO picture of glider and big wing

\subsection{Glider Competition Flight}

TODO analysis of the landing site


\section{Conclusions}
This paper presented a technology necessary to construct a rocket capable of achieving 10000 ft altitude and land safely on the ground. With the absence of control, the stability of the rocket was guaranteed by the extensive simulations done in 3 different simulators with one being developed by the team for the purpose of this project. The challenges with dynamic stability have also been explored and proper fins sizing done to dampen the oscillations. Throughout the subsystems presented, manufacturing process is also outlined to demonstrate the complete process from the idea through the conceptual design to actual prototype. 
The dual recovery process employed on the presented model composed of two parachutes and redundant electronics is capable of ensuring a safe landing even in presence of electronic faults. Custom made avionics, placed inside the nosecone of the rocket, provided necessary functionalities to control deployment of the payload and to log external parameters. It is designed to support further improvements such as airbrake or control system, possibly to be implemented in the following years. 
Second part of the paper presented the autonomous glider constructed and placed inside the rocket as a payload. It has been deployed at the apogee with the goal of controlled landing on a fixed location and sending data to the ground station. Concepts for a fleet of autonomous gliders flying in formation and landing on a predefined position was also presented together with challenges in terms of navigation and control. Throughout the project many tests have been performed on subsystem and system level to ensure compliance with requirements and to validate their functionalities, but due to the conciseness of this paper many results obtained are omitted. 

\section{Outlook}
The project presented above represents the first iteration and first prototype of both the rocket and the payload. The complete workload from the conceptual design to manufacturing has been done in the span of 7 months. Therefore, many of more advanced concepts which require time and thorough testing could not be implemented on this year's model. Some of those concepts are: implementation of the control system or the more simpler air brake system and the payload modifications to include the fleet of gliders. Even though they weren't implemented, many of these concepts have been researched and prototyped. Concept of the fleet of autonomous gliders has been presented in this paper, but also first iterations of the air brake systems have been developed and tested on one of the smaller models of the rocket. The test of this system was performed in a wind tunnel and subsequently the prototype was launched in Kaltbrunn, Switzerland on May the $20^{th} 2017$. Figure \ref{f:level1_airbrake} shows the smaller, so-called "Level 1" rocket with air brakes in a wind tunnel. 

  \begin{figure}[h!]
 	\centering
        \includegraphics[width=0.45\textwidth]{img/level1.png}
        \caption{Level 1 Rocket with Air Brakes in HEPIA Wind Tunnel}
        \label{f:level1_airbrake}
 \end{figure}

We expect some of these concepts to be further developed and implemented in the following years.


%Maybe talk about airbrakes, active control, etc.

\printglossary[nonumberlist] 

\bibliographystyle{ieeetr}
\bibliography{main}

\end{document}